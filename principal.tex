\documentclass{article}
\title{\textbf{PASCCO 2020}}
\author{\textbf{Elvis de Brito}}
\date{30 de outubro, 2020}
\usepackage{amsmath,amssymb,latexsym, amsthm} 
\usepackage{fullpage, indentfirst}
\usepackage[portuguese]{babel}
\usepackage{natbib}
\usepackage{textcomp}
\usepackage{tcolorbox}
\usepackage{chemfig}
\usepackage{color,graphicx}
\usepackage{siunitx}
\usepackage{hyperref}
\newtcolorbox{boxx}{colback=pink!20!white,colframe=black!100!black}
\newtcolorbox{carlos}{colback=black!1!white,colframe=black!75!black}


\title{{\fontfamily{put}\selectfont PASCCO 2020}
}
\author{\fontfamily{put}\selectfont {Elvis de Brito}
}
\date{30 de outubro, 2020}


\begin{document}


\maketitle

\begin{center}

\section*{\fontfamily{put}\selectfont Questionário
}

%Documento com indexação de respostas para o questionário da PASCCO 2020, total de 30 questões
%Aos corretores : agradeço a oportunidade

\end{center}


    \begin{boxx}
    \subsection*{Questão 1)}
    O fóton tem massa? Explique.
\end{boxx}
Resposta : 
 Não; a partir de análises em que relacionamos as partículas elementares e suas energias mínimas a seus alcances de interação, temos que :
 
 \begin{equation}
    \Delta E \approx m\cdot c^2 
 \end{equation}
 \begin{equation}
     \Delta E \cdot \Delta t \approx \hslash
 \end{equation}
 \begin{equation}
     \longrightarrow c \cdot \Delta t \approx \frac{\hslash}{m \cdot c}
 \end{equation}
 \begin{equation}
     V(r)=e^{-\frac{m\cdot c\cdot r}{\hslash}} \cdot \frac{1}{r}
 \end{equation}
 
     Portanto, temos que quanto maior a massa da partícula, menor seu alcance. Tendo o fóton massa nula:
     \begin{equation*}
       \longrightarrow  V(r) = 1 \cdot \frac{1}{r}
     \end{equation*}
     a equação corresponde ao seu longo alcance, o que corrobora com a predicção teórica.
     
     Apesar que, existem limites superiores para sua massa, advindo de experimentos que inferiram, inclusive, a lei de Coulomb, e conforme  \textit{Plimton and Lawton (1936)}, seu limite superior seria de $3,4\cdot10^{-44}\si{g}$. Esses limites de massas aferidas são, na verdade, uma suposição para aferimentos práticos, o que diverge da teoria. Atualmente a melhor aferência a cerca da massa do fóton é o limite superior de $10^{-55}\si{kg}$  


     
     
     \vspace{2mm}
    

\vspace{5mm}
\begin{boxx}
\subsection*{Questão 2)}
Qual é a equação de movimento do potencial vetor (A)? Explique.
\end{boxx}
  Para um campo vetorial, o potencial vetor é descrito conforme a equação \textcolor{blue}{(6)}, e dele obtém-se o resultado em \textcolor{blue}{(7)} :
\begin{equation}
    \nabla\cdot\mathbb{B}=0 
\end{equation}
\begin{equation}
    \mathbb{B}=\nabla\times\mathbb{A}
\end{equation}
\begin{equation}
    \Rightarrow \nabla\cdot(\nabla\times\mathbb{A})=0
\end{equation}




\vspace{5mm}
\begin{boxx}
\subsection*{Questão 3)}
Por que a equação de Klein-Gordon descreve a dinâmica de um campo escalar e não de um vetor?
\end{boxx}
Porque a equação de Klein-Gordon descreve os primórdios do estudo da Quântica, e sua primeira teorização é para partículas de spin 0, que precede as equações de Dirac para spin $\frac{1}{2}$. Assim, seus estudos se baseiam em campos quantizados e seus mais baixos estados de excitação, comportando-se como partículas massivas com spin 0, mas cargas positivas, negativas e neutras. 

\subsection*{Questão 4)}
\begin{boxx}
A equação de Dirac é invariante por qual tipo de simetria?
\end{boxx}
 As simetrias são:
\begin{itemize}
    \item simetria sob transformações de Lorentz (a equação de Dirac
é covariante sob as transformações de Lorentz);

    \item simetria sob U(1) global, no caso da equação livre;

    \item simetria sob U(1) local, no caso do elétron estar interagindo com
um campo eletromagnético extermo;

    \item simetria sob conjugação de carga, sob paridade e sob reversão temporal
(as 3 simetrias discretas); no final, simetria sob CPT.
\end{itemize}

\subsection*{Questão 5)}
\begin{boxx}
Por que a invariância de gauge local é importante?
\end{boxx}
 A simetria de gauge é fundamental para garantir que o fóton ou,
no caso geral, o bóson de gauge seja um puro spin-1. Se não houver
a simetria de gauge, a componente escalar do campo de gauge pode
se propagar e se acoplar e, assim, inviabilizar a teoria. Ou seja, a
simetria de gauge é importante para assegurar a transversalidade do
fóton.

\subsection*{Questão 6)}
\begin{boxx}
O que é um kink? Explique.
\end{boxx}

\subsection*{Questão 7)}
\begin{boxx}
O que você entende sobre a solução kink-antikink?
\end{boxx}

\subsection*{Questão 8)}
\begin{boxx}
O que é um buraco negro de Kerr?
\end{boxx}
 Formulado pelo físico Roy P. Kerr, é a construção teórica de um Buraco Negro com massa e rotação (momento angular).
  Conforme as equações :
  \begin{equation}
      ds^2 = - \bigg(\frac{\Delta-a^2\sin^2{\theta}}{\Sigma}\bigg)dt^2-\frac{2a\sin^2{\theta(r^2+a^2-\Delta)}}{\Sigma}dtd\phi + \bigg[\frac{(r^2+a^2)^2-\Delta a^2\sin^2{\theta}}{\Sigma}\bigg]\sin^2{\theta}d\phi^2 + \frac{\Sigma}{\Delta}dr^2 + \Sigma d \theta^2
  \end{equation}
  onde tem-se $\Sigma = r^2 + a^2\cos^2{\theta}$  e  $\Delta = r^2 + a^2 + e^2 - 2Mr$. Além de conter uma ergosfera.

 
\subsection*{Questão 9)}
\begin{boxx}
Fale um pouco sobre a termodinâmica de um buraco negro
\end{boxx}
Para o estudo termodinâmico de um buraco negro, Stephen Hawking formula ideias como em que a entropia é proporcional à area, e construções equivalentes às leis da termodinâmica para a proposta do fenômeno cósmico : 
\begin{itemize}
    \item Lei zero : $\kappa$ sendo constante, assim como $T$, além de que $\frac{\kappa}{2\pi}$
    
    \item Lei um : $dE = TdS-dW \longrightarrow dM = \kappa d A - dW$ 
    
    \item Lei dois : $S$ não decresce, como $A$ também não decresce
\end{itemize}
 
\subsection*{Questão 10)}
\begin{boxx}
O que é o processo de Penrose?
\end{boxx}
 É descrito por um físico inglês, que uma partícula, com uma determinada energia, caindo em um buraco, havendo sincronização no processo, ela poderia ejetar parte de sua massa/energia e assim sair com energia maior do que entrou.
 
\subsection*{Questão 11)}
\begin{boxx}
O que são: horizonte externo, horizonte interno, ergosfera?
\end{boxx}

\begin{itemize}
    \item Horizonte externo : limite de não retorno
    
    \item Horizonte interno : limite da predicção da dinâmica sem conhecimento do que acontece no buraco negro, também chamado de "horizonte de Cauchy". Teorizado como instável
    
    \item Ergosfera: região que circunda o horizonte de eventos e e o espaço-tempo é arrastado por sua rotação, ou seja, onde curvas com acelerações finitas não podem ficar paradas.
\end{itemize}

\subsection*{Questão 12)}
\begin{boxx}
 O próton pode decair em um pósitron? Explique. 
\end{boxx}
\subsection*{Questão 13)}
\begin{boxx}
O Higgs se acopla com o fóton ou glúons? Explique.
\end{boxx}
 O Bóson de Higgs, na teoria, não se acoplaria com nenhuma das partículas situadas no questionamento, a partir de que ambos não possuem massa, portanto não existe para os pares Higgs-fóton ou Higgs-glúon acoplamento efetivo.
 
\subsection*{Questão 14)}
\begin{boxx}
Existe um déficit no fluxo de neutrinos oriundos do Sol? Fale a respeito.
\end{boxx}

 Sim; conforme o avanço da formulação das partículas fundamentais e a descoberta da existência dos neutrinos, viu-se que o Sol, dentro de seu modelo padrão estelar, emitia além de fótons também os neutrinos-eletrônicos $(\nu_e)$; dois exemplos de reações recorrentes e  que correspondem a 87\% do fluxo dessas "novas"\space partículas que chegam na Terra (86\%+1\% respectivamente) são :

\begin{center}
    \schemestart
       p + p \arrow{->} d + e$^+$ + $\nu_e$ ($E \leq 0.42 \si{\MeV}$)
    \schemestop
\end{center}

\begin{center}
    \schemestart
      p + e$^-$ + p \arrow{->} d + $\nu_e$ ($E = 1.44 \si{\MeV}$)
    \schemestop
\end{center}
 
 Esses neutrinos-eletrônicos atravessam a terra e seguem além-Universo. Ao teorizar-se a quantidade deles que adviam do Sol, percebeu-se que os cálculos não conferiam com os dados experimentais - os neutrinos vindos do sol eram apenas $\frac{1}{3}$ do que o esperado, o que ficou conhecido como "Problema dos Neutrinos Solares"\space Após questionamentos a cerca das teorias envolvidas (Modelo padrão Solar e o Modelo Padrão de Partículas elementares) viu-se que elas estavam corretas, e o Problema dos Neutrinos Solares precisava de uma explicação. A teoria atualmente aceita é a de Oscilação entre Neutrinos : 
 
 \begin{center}
    \schemestart
    $\nu_e$ \arrow{<=>} $\nu_{\tau}$ \arrow{<=>} $\nu_\mu$ 
    \schemestop
\end{center}

Portanto, viu-se que o fato dos detectores estarem configurados para contabilizarem o neutrino-eletrônico, eles não constatavam suas transmutações para o neutrino-tau e o neutrino-múon.
\subsection*{Questão 15)}
\begin{boxx}
Por que os léptons carregados não oscilam?
\end{boxx}
 Porque a previsão de oscilação(que começa com os neutrinos) é explicada pela fórmula :
 
\begin{equation}
    P(l\to l', x) = \sum_m U_{lm}^2\cdot U_{l'm}^2 + \sum_{m'\neq m} U_{lm} \cdot U_{l'm} \cdot U_{lm'} \cdot U_{l'm'} \cdot \cos{\bigg( 2\pi\frac{x}{L_{m',m}} \bigg) }
\end{equation}
, onde $L_{m',m} = 2\pi\cdot\frac{2P_v}{|M_m^2-M_m'^2|}$ e $U \equiv \text{matriz de mistura}$; portanto observa-se que para massas muito díspares $M_m^2 >> M_m'^2$ ou $M_m^2 << M_m'^2$ perdemos $L_{m',m}$ porque tende a 0, logo também a oscilação.
 Observemos que: 
  \begin{itemize}
     \item $M_{e^-} \approx 0.5 \si{MeV/c^2}$
      
     \item $M_{\tau^-} \approx 1 780 \si{MeV/c^2}$
      
     \item  $M_\mu^- \approx 105.7 \si{MeV/c^2}$
  \end{itemize}
   As massas dos léptons são muito distintas entre si, em ordens de grandeza, portanto $L_{m',m} \to 0$ o que corrobora a pouca probabilidade da oscilação dos léptons. 


\subsection*{Questão 16)}
\begin{boxx}

Qual o operador efetivo que nos guia para gerar massa de
Majorana para os neutrinos?
\end{boxx}

 Um operador efetivo de dimensão-5, da forma da equação \textcolor{blue}{(12)}:
 \begin{equation}
    \frac{y}{\Lambda}(\Bar{L}^c  \Tilde{H})(\Tilde{H}^T L)
 \end{equation}
 
 
 \begin{equation}
     \langle H \rangle = \frac{v}{\sqrt{2}}
 \end{equation}
 
 Em seguida dos cálculos, as equações \textcolor{blue}{(12)} e \textcolor{blue}{(13)} dão forma às equações a seguir : 
 \begin{equation}
     \Longrightarrow \frac{y}{\Lambda} \bigg[         \begin{pmatrix}             \Bar{\nu}_l^c &             \Bar{e}_l^c       \end{pmatrix}   \begin{pmatrix} \frac{v}{\sqrt{2}}\\0       \end{pmatrix} \bigg] \cdot \bigg[ \begin{pmatrix} \frac{v}{\sqrt{2}} & 0   \end{pmatrix} 
         \begin{pmatrix} 
         \nu_l \\ e_l
         \end{pmatrix} 
         \bigg]
 \end{equation}
 
 \begin{equation}
     \frac{1}{2}\lambda \frac{v^2}{\Lambda}\Bar{\nu}_l^c \nu_l
 \end{equation}

 \begin{equation}
\Longrightarrow 
     M = \frac{1}{2}\lambda \frac{v^2}{\Lambda}
 \end{equation}
  A operador efetivo guia-nos em gerar mecanismos de Seesaw para gerar massa de Majorana para os neutrinos.

\subsection*{Questão 17)}
\begin{boxx}
O que é o duplo decaimento beta? Quais são os isótopos usados nos experimentos Exo, Kamlanda-Zen, e Heidelberg-Moscow?
\end{boxx}
 É o decaimento espontâneo e raro dentro dos núcleos atômicos, onde se libera partículas $\beta^-$ ou $\beta^+$ na forma $\beta^-\beta^-$ ou $\beta^+\beta^+$.
 
  O isótopo usado no experimento EXO-200(Enriched Xenon Observatory) é o $^{136}$Xe; no experimento KAMLAND-Zen usa-se os isótopos $^{136}$Xe e $^{134}$Xe, cada isótopo correspondendo a 90\% e 8,9\%, respectivamente; para o experimento Heidelberg-Moscow usa-se $^{76}$Ge.  
  
  
\subsection*{Questão 18)}
\begin{boxx}
O que é o teorema de Goldstone?
\end{boxx}
 O estado de vácuo tem uma quebra de simetria U(1) global. Goldstone, Salam e Weinberg precedendo os estudos sobre os bósons de Nambu-Goldstone, determinam que em uma teoria invariante por Lorentz, a quebra de simetria espontânea resulta na necessidade de uma partícula não-massiva no espectro.
 
  $\Longrightarrow$ Se uma simetria contínua é espontaneamente quebrada, campos sem massa, conhecidos como bósons de Nambu-Goldstone, aparecerão.
  
\subsection*{Questão 19)}
\begin{boxx}
Em 2012 o Higgs foi descoberto. Quais os valores da massa do Higgs medidas pelos experimentos CMS e ATLAS na época?
\end{boxx}
 Em julho de 2012, o CMS e o ATLAS aferiram para uma partícula, idêntica a qual seria o Bóson de Higgs, a massa entre 125\si{GeV/c^2} e 126\si{GeV/c^2}; posteriormente confirmou-se que era o próprio Bóson de Higgs teorizado pelo modelo padrão. 
 
\subsection*{Questão 20)}
\begin{boxx}
Que tipo de buracos negros podem representar uma componente de Matéria Escura?
\end{boxx}

\subsection*{Questão 21)}
\begin{boxx}
Explique os métodos de detecção de Matéria Escura. 
\end{boxx}

\subsection*{Questão 22)}
\begin{boxx}
O que é a concordância cósmica?
\end{boxx}



\subsection*{Questão 23)}
\begin{boxx}
O que é a Lei de Hubble?
\end{boxx}

 A Lei de Hubble é a formulação do astrônomo Edwin Powell Hubble para o afastamento das galáxias em relação à Via Láctea, que se dá  relacionando a velocidade pela distância :
 
 $$ v = H_0 \cdot d $$
 
 ou sua forma generalizada : 
 
 $$ v_i = \Sigma_i (\Vec{d}{_i} + \Vec{\lambda}_i) $$


\subsection*{Questão 24)}
\begin{boxx}
Por que a CMB tem um espectro de corpo negro?
\end{boxx}

\subsection*{Questão 25}
\begin{boxx}
Para qual ângulo entre dois pontos no espaço a anisotropia da CMB é máxima?
\end{boxx}

\subsection*{Questão 26)}
\begin{boxx}
Com o CAMB podemos estudar Neff? Qual o efeito disto na matéria?
\end{boxx}

\subsection*{Questão 27)}
\begin{boxx}
O que é escala de corte no espectro de potência?
\end{boxx}

\subsection*{Questão 28)}
\begin{boxx}
Que tipo de hidrogênio é usado para sondar o universo? Qual é a tendência?
\end{boxx}

\subsection*{Questão 29)}
\begin{boxx}
O BINGO almeja sondar quais parâmetros?
\end{boxx}
BINGO é um rádio telescópio. Sua busca é medir Oscilações Acústicas de Bárions (BAO) utilizando emissão de rádio de Hidrogênio. 

\subsection*{Questão 30)}
\begin{boxx}
O que é a linha de 21cm?
\end{boxx}
 A radiação de 21 cm do hidrogênio atômico foi
prevista teoricamente pelo dinamarquês H. C. van de Hulst,
quando propôs, em 1944, que o átomo de hidrogênio emitiria
uma radiação nesse comprimento de onda como resultado
da variação do spin do elétron.

Por serem o próton e o elétron partículas com cargas elétricas em rotação, eles criam campos magnéticos que interagem,
de forma que o estado de $less-energy$ é com spins
antiparalelos, e o de $higher-energy$ com spins paralelos. A
diferença de energia entre os dois níveis é de $hv = 6 x 10^-6$ eV,
o que corresponde a frequência de 1.420,4 MHz. Assim,
a transição entre esses dois níveis de estrutura origina uma linha de comprimento de onda de $\lambda =c/v$ 21,049 cm, na faixa de rádio.

%\subsection{Questão 1}

%\subsection{Questão 1}

%\section{Respostas}

%\begin{enumerate}
 
%     \[  \left( \frac{243}{123} \right) \]
    
%\end{enumerate}


%\subsection{Questão 1}



%\begin{figure}[h!]
%\centering
%\includegraphics[scale=3.0]{universe}
%\caption{The Universe}
%\label{fig:universe}
%\end{figure}

%\begin{figure}[h!]

%\includegraphics[scale=0.5]{laranja-01}
%\caption{Laranja}
%\label{fig:universe}
%\end{figure}
\begin{center}
    \includegraphics[scale=0.3]{Standard Elemental particles theory.png}
\end{center}


\newpage
\begin{center}
 \section*{Conclusão}
 Agradeço ao professor José Adballa Helayël, que me acolheu como aluno, e a Professora Patrícia, me auxiliaram nas direcções de minhas respostas.
Agradeço ao(s) corretor(es), uma ótima oportunidade de aprendizado.

Espero nos encontrarmos pela vida.

\vspace{5mm}

  {\fontfamily{qcr}\selectfont
"Tudo no mundo começou com um sim. Uma molécula disse sim a outra molécula e nasceu a vida. Mas antes da pré-história havia a pré-história da pré-história e havia o nunca e havia o sim. Sempre houve. Não sei o quê, mas sei que o universo jamais começou."

\vspace{3mm}
Lispector, Clarice
}
   
\end{center}

\newpage
\section*{\textit{Referências}}
\begin{enumerate}
    \item{Photon and Graviton Mass Limits}
    
    \textcolor{blue}{https://arxiv.org/pdf/0809.1003.pdf}
    
    \item{PASCCO : Modelo Padrão de Física de Partículas Aula 1- Paulo Silva - UFPB}
    
   \textcolor{blue}{https://youtu.be/15FdYdGiArs?t=1006}
   
   \item{O Fóton tem Massa? Faça Pesquisa Científica com a Lei de Coulomb! - Manoel Messias - UFMA}
   \textcolor{blue}{https://www.youtube.com/watch?v=7VJdRF2nIC0}
   
   \item{Produção de Fotinos e Gluínos nas Extensões Supersimétricas da Eletrodinâmica Quântica e da Cromodinâmica Quântica - Espíndola, Danusa Bueno}
   
   \textcolor{blue}{https://lume.ufrgs.br/handle/10183/31009}
   
   \item{Experimental tests of Coulomb's Law and the photon rest mass - Liang-Cheng Tu and Jun Luo}
   
 \textcolor{blue}{https://bit.ly/3e4UBmY} 
 
   \item{PASCCO: Física de Neutrinos Aula 1 - Carlos Pires - UFPB}

\textcolor{blue}{https://www.youtube.com/watch?v=zfPT5XcBhhQ&feature=emb_titl}

   \item{Compreendendo a oscilação dos neutrinos - Gustavo do A. Valdiviesso e Marcelo M. Guzzo}
 
   
\textcolor{blue}{http://www.sbfisica.org.br/rbef/pdf/v27_495.pdf}
   
   
   \item{Neutrinos Solares} 
   
   \textcolor{blue}{http://propg.ufabc.edu.br/mnpef-sites/neutrinos/index.php/fontes/solares/}
   
   \item{Dublo Decaimento Beta Aula 1 e 2 - Diego Cogollo - UFCG}
   
   \textcolor{blue}{https://www.youtube.com/watch?v=fafHVkzPvN4&feature=emb_title}
   
   \item{Os Átomos São Eternos? (O Decaimento de Prótons) - Pedro Loos}
   
   \textcolor{blue}{https://www.youtube.com/watch?v=Apz2y3kZxRY}
   
   \item A Lei de Hubble e a Homogeneidade do Universo - Fernando Kokubun
   
   \textcolor{blue}{http://sbfisica.org.br/rbef/pdf/v21_311.pdf}
   
   \item Vinculando a fase de violação de CP de neutrinos de Majorana na era de precisão da Cosmologia e dos experimentos de duplo decaimento Beta - Alexandre Argüello Quiroga
   
   \textcolor{blue}{https://bit.ly/3eavYVY} 
   
   \item The HEIDELBERG-MOSCOW EXPERIMENT - H.V.Klapdor-Kleingrothaus

   \textcolor{blue}{https://bit.ly/2TEVtW2}
   
   \item Uma Introdução à Teoria Quântica de Campos: Quebra
   Espontânea de Simetria - Heitor do Amaral Jurkovich
  
   \textcolor{blue}{https://bit.ly/3jEXrjD}
   
   \item O telescópio BINGO: a nova janela de 21cm para exploração do universo escuro e outras questões astrofísicas 
   
   \textcolor{blue}{https://bit.ly/3jKzyr8}
   
   \item Aula 22: Meio interestelar - Maria de Fátima Oliveira Saraiva, Kepler de Souza Oliveira Filho e Alexei Machado Müller

    
   
   \textcolor{blue}{https://bit.ly/384di9c}
   
   \item PASCCO: Buracos Negros Aula 1 - Bruno Carneiro -UFPE
   
   \textcolor{blue}{https://www.youtube.com/watch?v=PAOu_cqbq-4&feature=emb_title}
   
   \item PASCCO : Buracos Negros Aula 2 - Bruno Carneiro -UFPE
   
   \textcolor{blue}{https://www.youtube.com/watch?v=GkhAjvrWIeU&feature=emb_title}
   
   \item Mecanismo de Higgs - Luciano Abreu - Aula 2
   
  \textcolor{blue}{https://www.youtube.com/watch?v=tgMJXiZ_PqU}
   
\end{enumerate}
\end{document}
